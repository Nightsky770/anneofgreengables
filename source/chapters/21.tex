%!TeX root=../annetop.tex
\chapter{A New Departure in Flavourings}

\lettrine[ante=“,lines=4]{D}{ear} me, there is nothing but meetings and partings in this world, as Mrs. Lynde says,” remarked Anne plaintively, putting her slate and books down on the kitchen table on the last day of June and wiping her red eyes with a very damp handkerchief. »Wasn't it fortunate, Marilla, that I took an extra handkerchief to school today? I had a presentiment that it would be needed.«

»I never thought you were so fond of Mr. Phillips that you'd require two handkerchiefs to dry your tears just because he was going away,« said Marilla.

»I don't think I was crying because I was really so very fond of him,« reflected Anne. »I just cried because all the others did. It was Ruby Gillis started it. Ruby Gillis has always declared she hated Mr. Phillips, but just as soon as he got up to make his farewell speech she burst into tears. Then all the girls began to cry, one after the other. I tried to hold out, Marilla. I tried to remember the time Mr. Phillips made me sit with Gil—with a boy; and the time he spelled my name without an »e« on the blackboard; and how he said I was the worst dunce he ever saw at geometry and laughed at my spelling; and all the times he had been so horrid and sarcastic; but somehow I couldn't, Marilla, and I just had to cry too. Jane Andrews has been talking for a month about how glad she'd be when Mr. Phillips went away and she declared she'd never shed a tear. Well, she was worse than any of us and had to borrow a handkerchief from her brother—of course the boys didn't cry—because she hadn't brought one of her own, not expecting to need it. Oh, Marilla, it was heartrending. Mr. Phillips made such a beautiful farewell speech beginning, »The time has come for us to part.« It was very affecting. And he had tears in his eyes too, Marilla. Oh, I felt dreadfully sorry and remorseful for all the times I'd talked in school and drawn pictures of him on my slate and made fun of him and Prissy. I can tell you I wished I'd been a model pupil like Minnie Andrews. She hadn't anything on her conscience. The girls cried all the way home from school. Carrie Sloane kept saying every few minutes, »The time has come for us to part,« and that would start us off again whenever we were in any danger of cheering up. I do feel dreadfully sad, Marilla. But one can't feel quite in the depths of despair with two months' vacation before them, can they, Marilla? And besides, we met the new minister and his wife coming from the station. For all I was feeling so bad about Mr. Phillips going away I couldn't help taking a little interest in a new minister, could I? His wife is very pretty. Not exactly regally lovely, of course—it wouldn't do, I suppose, for a minister to have a regally lovely wife, because it might set a bad example. Mrs. Lynde says the minister's wife over at Newbridge sets a very bad example because she dresses so fashionably. Our new minister's wife was dressed in blue muslin with lovely puffed sleeves and a hat trimmed with roses. Jane Andrews said she thought puffed sleeves were too worldly for a minister's wife, but I didn't make any such uncharitable remark, Marilla, because I know what it is to long for puffed sleeves. Besides, she's only been a minister's wife for a little while, so one should make allowances, shouldn't they? They are going to board with Mrs. Lynde until the manse is ready.«

If Marilla, in going down to Mrs. Lynde's that evening, was actuated by any motive save her avowed one of returning the quilting frames she had borrowed the preceding winter, it was an amiable weakness shared by most of the Avonlea people. Many a thing Mrs. Lynde had lent, sometimes never expecting to see it again, came home that night in charge of the borrowers thereof. A new minister, and moreover a minister with a wife, was a lawful object of curiosity in a quiet little country settlement where sensations were few and far between.

Old Mr. Bentley, the minister whom Anne had found lacking in imagination, had been pastor of Avonlea for eighteen years. He was a widower when he came, and a widower he remained, despite the fact that gossip regularly married him to this, that, or the other one, every year of his sojourn. In the preceding February he had resigned his charge and departed amid the regrets of his people, most of whom had the affection born of long intercourse for their good old minister in spite of his shortcomings as an orator. Since then the Avonlea church had enjoyed a variety of religious dissipation in listening to the many and various candidates and »supplies« who came Sunday after Sunday to preach on trial. These stood or fell by the judgment of the fathers and mothers in Israel; but a certain small, red-haired girl who sat meekly in the corner of the old Cuthbert pew also had her opinions about them and discussed the same in full with Matthew, Marilla always declining from principle to criticize ministers in any shape or form.

»I don't think Mr. Smith would have done, Matthew« was Anne's final summing up. »Mrs. Lynde says his delivery was so poor, but I think his worst fault was just like Mr. Bentley's—he had no imagination. And Mr. Terry had too much; he let it run away with him just as I did mine in the matter of the Haunted Wood. Besides, Mrs. Lynde says his theology wasn't sound. Mr. Gresham was a very good man and a very religious man, but he told too many funny stories and made the people laugh in church; he was undignified, and you must have some dignity about a minister, mustn't you, Matthew? I thought Mr. Marshall was decidedly attractive; but Mrs. Lynde says he isn't married, or even engaged, because she made special inquiries about him, and she says it would never do to have a young unmarried minister in Avonlea, because he might marry in the congregation and that would make trouble. Mrs. Lynde is a very farseeing woman, isn't she, Matthew? I'm very glad they've called Mr. Allan. I liked him because his sermon was interesting and he prayed as if he meant it and not just as if he did it because he was in the habit of it. Mrs. Lynde says he isn't perfect, but she says she supposes we couldn't expect a perfect minister for seven hundred and fifty dollars a year, and anyhow his theology is sound because she questioned him thoroughly on all the points of doctrine. And she knows his wife's people and they are most respectable and the women are all good housekeepers. Mrs. Lynde says that sound doctrine in the man and good housekeeping in the woman make an ideal combination for a minister's family.«

The new minister and his wife were a young, pleasant-faced couple, still on their honeymoon, and full of all good and beautiful enthusiasms for their chosen lifework. Avonlea opened its heart to them from the start. Old and young liked the frank, cheerful young man with his high ideals, and the bright, gentle little lady who assumed the mistress-ship of the manse. With Mrs. Allan Anne fell promptly and wholeheartedly in love. She had discovered another kindred spirit.

»Mrs. Allan is perfectly lovely,« she announced one Sunday afternoon. »She's taken our class and she's a splendid teacher. She said right away she didn't think it was fair for the teacher to ask all the questions, and you know, Marilla, that is exactly what I've always thought. She said we could ask her any question we liked and I asked ever so many. I'm good at asking questions, Marilla.«

»I believe you,« was Marilla's emphatic comment.

»Nobody else asked any except Ruby Gillis, and she asked if there was to be a Sunday-school picnic this summer. I didn't think that was a very proper question to ask because it hadn't any connection with the lesson—the lesson was about Daniel in the lions' den—but Mrs. Allan just smiled and said she thought there would be. Mrs. Allan has a lovely smile; she has such exquisite dimples in her cheeks. I wish I had dimples in my cheeks, Marilla. I'm not half so skinny as I was when I came here, but I have no dimples yet. If I had perhaps I could influence people for good. Mrs. Allan said we ought always to try to influence other people for good. She talked so nice about everything. I never knew before that religion was such a cheerful thing. I always thought it was kind of melancholy, but Mrs. Allan's isn't, and I'd like to be a Christian if I could be one like her. I wouldn't want to be one like Mr. Superintendent Bell.«

»It's very naughty of you to speak so about Mr. Bell,« said Marilla severely. »Mr. Bell is a real good man.«

»Oh, of course he's good,« agreed Anne, »but he doesn't seem to get any comfort out of it. If I could be good I'd dance and sing all day because I was glad of it. I suppose Mrs. Allan is too old to dance and sing and of course it wouldn't be dignified in a minister's wife. But I can just feel she's glad she's a Christian and that she'd be one even if she could get to heaven without it.«

»I suppose we must have Mr. and Mrs. Allan up to tea someday soon,« said Marilla reflectively. »They've been most everywhere but here. Let me see. Next Wednesday would be a good time to have them. But don't say a word to Matthew about it, for if he knew they were coming he'd find some excuse to be away that day. He'd got so used to Mr. Bentley he didn't mind him, but he's going to find it hard to get acquainted with a new minister, and a new minister's wife will frighten him to death.«

»I'll be as secret as the dead,« assured Anne. »But oh, Marilla, will you let me make a cake for the occasion? I'd love to do something for Mrs. Allan, and you know I can make a pretty good cake by this time.«

»You can make a layer cake,« promised Marilla.

Monday and Tuesday great preparations went on at Green Gables. Having the minister and his wife to tea was a serious and important undertaking, and Marilla was determined not to be eclipsed by any of the Avonlea housekeepers. Anne was wild with excitement and delight. She talked it all over with Diana Tuesday night in the twilight, as they sat on the big red stones by the Dryad's Bubble and made rainbows in the water with little twigs dipped in fir balsam.

»Everything is ready, Diana, except my cake which I'm to make in the morning, and the baking-powder biscuits which Marilla will make just before teatime. I assure you, Diana, that Marilla and I have had a busy two days of it. It's such a responsibility having a minister's family to tea. I never went through such an experience before. You should just see our pantry. It's a sight to behold. We're going to have jellied chicken and cold tongue. We're to have two kinds of jelly, red and yellow, and whipped cream and lemon pie, and cherry pie, and three kinds of cookies, and fruit cake, and Marilla's famous yellow plum preserves that she keeps especially for ministers, and pound cake and layer cake, and biscuits as aforesaid; and new bread and old both, in case the minister is dyspeptic and can't eat new. Mrs. Lynde says ministers are dyspeptic, but I don't think Mr. Allan has been a minister long enough for it to have had a bad effect on him. I just grow cold when I think of my layer cake. Oh, Diana, what if it shouldn't be good! I dreamed last night that I was chased all around by a fearful goblin with a big layer cake for a head.«

»It'll be good, all right,« assured Diana, who was a very comfortable sort of friend. »I'm sure that piece of the one you made that we had for lunch in Idlewild two weeks ago was perfectly elegant.«

»Yes; but cakes have such a terrible habit of turning out bad just when you especially want them to be good,« sighed Anne, setting a particularly well-balsamed twig afloat. »However, I suppose I shall just have to trust to Providence and be careful to put in the flour. Oh, look, Diana, what a lovely rainbow! Do you suppose the dryad will come out after we go away and take it for a scarf?«

»You know there is no such thing as a dryad,« said Diana. Diana's mother had found out about the Haunted Wood and had been decidedly angry over it. As a result Diana had abstained from any further imitative flights of imagination and did not think it prudent to cultivate a spirit of belief even in harmless dryads.

»But it's so easy to imagine there is,« said Anne. »Every night before I go to bed, I look out of my window and wonder if the dryad is really sitting here, combing her locks with the spring for a mirror. Sometimes I look for her footprints in the dew in the morning. Oh, Diana, don't give up your faith in the dryad!«

Wednesday morning came. Anne got up at sunrise because she was too excited to sleep. She had caught a severe cold in the head by reason of her dabbling in the spring on the preceding evening; but nothing short of absolute pneumonia could have quenched her interest in culinary matters that morning. After breakfast she proceeded to make her cake. When she finally shut the oven door upon it she drew a long breath.

»I'm sure I haven't forgotten anything this time, Marilla. But do you think it will rise? Just suppose perhaps the baking powder isn't good? I used it out of the new can. And Mrs. Lynde says you can never be sure of getting good baking powder nowadays when everything is so adulterated. Mrs. Lynde says the Government ought to take the matter up, but she says we'll never see the day when a Tory Government will do it. Marilla, what if that cake doesn't rise?«

»We'll have plenty without it« was Marilla's unimpassioned way of looking at the subject.

The cake did rise, however, and came out of the oven as light and feathery as golden foam. Anne, flushed with delight, clapped it together with layers of ruby jelly and, in imagination, saw Mrs. Allan eating it and possibly asking for another piece!

»You'll be using the best tea set, of course, Marilla,« she said. »Can I fix the table with ferns and wild roses?«

»I think that's all nonsense,« sniffed Marilla. »In my opinion it's the eatables that matter and not flummery decorations.«

»Mrs. Barry had her table decorated,« said Anne, who was not entirely guiltless of the wisdom of the serpent, »and the minister paid her an elegant compliment. He said it was a feast for the eye as well as the palate.«

»Well, do as you like,« said Marilla, who was quite determined not to be surpassed by Mrs. Barry or anybody else. »Only mind you leave enough room for the dishes and the food.«

Anne laid herself out to decorate in a manner and after a fashion that should leave Mrs. Barry's nowhere. Having abundance of roses and ferns and a very artistic taste of her own, she made that tea table such a thing of beauty that when the minister and his wife sat down to it they exclaimed in chorus over it loveliness.

»It's Anne's doings,« said Marilla, grimly just; and Anne felt that Mrs. Allan's approving smile was almost too much happiness for this world.

Matthew was there, having been inveigled into the party only goodness and Anne knew how. He had been in such a state of shyness and nervousness that Marilla had given him up in despair, but Anne took him in hand so successfully that he now sat at the table in his best clothes and white collar and talked to the minister not uninterestingly. He never said a word to Mrs. Allan, but that perhaps was not to be expected.

All went merry as a marriage bell until Anne's layer cake was passed. Mrs. Allan, having already been helped to a bewildering variety, declined it. But Marilla, seeing the disappointment on Anne's face, said smilingly:

»Oh, you must take a piece of this, Mrs. Allan. Anne made it on purpose for you.«

»In that case I must sample it,« laughed Mrs. Allan, helping herself to a plump triangle, as did also the minister and Marilla.

Mrs. Allan took a mouthful of hers and a most peculiar expression crossed her face; not a word did she say, however, but steadily ate away at it. Marilla saw the expression and hastened to taste the cake.

»Anne Shirley!« she exclaimed, »what on earth did you put into that cake?«

»Nothing but what the recipe said, Marilla,« cried Anne with a look of anguish. »Oh, isn't it all right?«

»All right! It's simply horrible. Mr. Allan, don't try to eat it. Anne, taste it yourself. What flavouring did you use?«

»Vanilla,« said Anne, her face scarlet with mortification after tasting the cake. »Only vanilla. Oh, Marilla, it must have been the baking powder. I had my suspicions of that bak\longdash«

»Baking powder fiddlesticks! Go and bring me the bottle of vanilla you used.«

Anne fled to the pantry and returned with a small bottle partially filled with a brown liquid and labeled yellowly, »Best Vanilla.«

Marilla took it, uncorked it, smelled it.

»Mercy on us, Anne, you've flavoured that cake with Anodyne Liniment. I broke the liniment bottle last week and poured what was left into an old empty vanilla bottle. I suppose it's partly my fault—I should have warned you—but for pity's sake why couldn't you have smelled it?«

Anne dissolved into tears under this double disgrace.

»I couldn't—I had such a cold!« and with this she fairly fled to the gable chamber, where she cast herself on the bed and wept as one who refuses to be comforted.

Presently a light step sounded on the stairs and somebody entered the room.

»Oh, Marilla,« sobbed Anne, without looking up, »I'm disgraced forever. I shall never be able to live this down. It will get out—things always do get out in Avonlea. Diana will ask me how my cake turned out and I shall have to tell her the truth. I shall always be pointed at as the girl who flavoured a cake with anodyne liniment. Gil—the boys in school will never get over laughing at it. Oh, Marilla, if you have a spark of Christian pity don't tell me that I must go down and wash the dishes after this. I'll wash them when the minister and his wife are gone, but I cannot ever look Mrs. Allan in the face again. Perhaps she'll think I tried to poison her. Mrs. Lynde says she knows an orphan girl who tried to poison her benefactor. But the liniment isn't poisonous. It's meant to be taken internally—although not in cakes. Won't you tell Mrs. Allan so, Marilla?«

»Suppose you jump up and tell her so yourself,« said a merry voice.

Anne flew up, to find Mrs. Allan standing by her bed, surveying her with laughing eyes.

»My dear little girl, you mustn't cry like this,« she said, genuinely disturbed by Anne's tragic face. »Why, it's all just a funny mistake that anybody might make.«

»Oh, no, it takes me to make such a mistake,« said Anne forlornly. »And I wanted to have that cake so nice for you, Mrs. Allan.«

»Yes, I know, dear. And I assure you I appreciate your kindness and thoughtfulness just as much as if it had turned out all right. Now, you mustn't cry any more, but come down with me and show me your flower garden. Miss Cuthbert tells me you have a little plot all your own. I want to see it, for I'm very much interested in flowers.«

Anne permitted herself to be led down and comforted, reflecting that it was really providential that Mrs. Allan was a kindred spirit. Nothing more was said about the liniment cake, and when the guests went away Anne found that she had enjoyed the evening more than could have been expected, considering that terrible incident. Nevertheless, she sighed deeply.

»Marilla, isn't it nice to think that tomorrow is a new day with no mistakes in it yet?«

»I'll warrant you'll make plenty in it,« said Marilla. »I never saw your beat for making mistakes, Anne.«

»Yes, and well I know it,« admitted Anne mournfully. »But have you ever noticed one encouraging thing about me, Marilla? I never make the same mistake twice.«

»I don't know as that's much benefit when you're always making new ones.«

»Oh, don't you see, Marilla? There must be a limit to the mistakes one person can make, and when I get to the end of them, then I'll be through with them. That's a very comforting thought.«

»Well, you'd better go and give that cake to the pigs,« said Marilla. »It isn't fit for any human to eat, not even Jerry Boute.«