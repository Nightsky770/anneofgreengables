%!TeX root=../annetop.tex
\chapter{Where the Brook and River Meet}

\lettrine[lines=4]{A}{nne} had her »good« summer and enjoyed it wholeheartedly. She and Diana fairly lived outdoors, revelling in all the delights that Lover's Lane and the Dryad's Bubble and Willowmere and Victoria Island afforded. Marilla offered no objections to Anne's gypsyings. The Spencervale doctor who had come the night Minnie May had the croup met Anne at the house of a patient one afternoon early in vacation, looked her over sharply, screwed up his mouth, shook his head, and sent a message to Marilla Cuthbert by another person. It was:

»Keep that redheaded girl of yours in the open air all summer and don't let her read books until she gets more spring into her step.«

This message frightened Marilla wholesomely. She read Anne's death warrant by consumption in it unless it was scrupulously obeyed. As a result, Anne had the golden summer of her life as far as freedom and frolic went. She walked, rowed, berried, and dreamed to her heart's content; and when September came she was bright-eyed and alert, with a step that would have satisfied the Spencervale doctor and a heart full of ambition and zest once more.

»I feel just like studying with might and main,« she declared as she brought her books down from the attic. »Oh, you good old friends, I'm glad to see your honest faces once more—yes, even you, geometry. I've had a perfectly beautiful summer, Marilla, and now I'm rejoicing as a strong man to run a race, as Mr. Allan said last Sunday. Doesn't Mr. Allan preach magnificent sermons? Mrs. Lynde says he is improving every day and the first thing we know some city church will gobble him up and then we'll be left and have to turn to and break in another green preacher. But I don't see the use of meeting trouble halfway, do you, Marilla? I think it would be better just to enjoy Mr. Allan while we have him. If I were a man I think I'd be a minister. They can have such an influence for good, if their theology is sound; and it must be thrilling to preach splendid sermons and stir your hearers' hearts. Why can't women be ministers, Marilla? I asked Mrs. Lynde that and she was shocked and said it would be a scandalous thing. She said there might be female ministers in the States and she believed there was, but thank goodness we hadn't got to that stage in Canada yet and she hoped we never would. But I don't see why. I think women would make splendid ministers. When there is a social to be got up or a church tea or anything else to raise money the women have to turn to and do the work. I'm sure Mrs. Lynde can pray every bit as well as Superintendent Bell and I've no doubt she could preach too with a little practise.«

»Yes, I believe she could,« said Marilla dryly. »She does plenty of unofficial preaching as it is. Nobody has much of a chance to go wrong in Avonlea with Rachel to oversee them.«

»Marilla,« said Anne in a burst of confidence, »I want to tell you something and ask you what you think about it. It has worried me terribly—on Sunday afternoons, that is, when I think specially about such matters. I do really want to be good; and when I'm with you or Mrs. Allan or Miss Stacy I want it more than ever and I want to do just what would please you and what you would approve of. But mostly when I'm with Mrs. Lynde I feel desperately wicked and as if I wanted to go and do the very thing she tells me I oughtn't to do. I feel irresistibly tempted to do it. Now, what do you think is the reason I feel like that? Do you think it's because I'm really bad and unregenerate?«

Marilla looked dubious for a moment. Then she laughed.

»If you are I guess I am too, Anne, for Rachel often has that very effect on me. I sometimes think she'd have more of an influence for good, as you say yourself, if she didn't keep nagging people to do right. There should have been a special commandment against nagging. But there, I shouldn't talk so. Rachel is a good Christian woman and she means well. There isn't a kinder soul in Avonlea and she never shirks her share of work.«

»I'm very glad you feel the same,« said Anne decidedly. »It's so encouraging. I shan't worry so much over that after this. But I dare say there'll be other things to worry me. They keep coming up new all the time—things to perplex you, you know. You settle one question and there's another right after. There are so many things to be thought over and decided when you're beginning to grow up. It keeps me busy all the time thinking them over and deciding what is right. It's a serious thing to grow up, isn't it, Marilla? But when I have such good friends as you and Matthew and Mrs. Allan and Miss Stacy I ought to grow up successfully, and I'm sure it will be my own fault if I don't. I feel it's a great responsibility because I have only the one chance. If I don't grow up right I can't go back and begin over again. I've grown two inches this summer, Marilla. Mr. Gillis measured me at Ruby's party. I'm so glad you made my new dresses longer. That dark-green one is so pretty and it was sweet of you to put on the flounce. Of course I know it wasn't really necessary, but flounces are so stylish this fall and Josie Pye has flounces on all her dresses. I know I'll be able to study better because of mine. I shall have such a comfortable feeling deep down in my mind about that flounce.«

»It's worth something to have that,« admitted Marilla.

Miss Stacy came back to Avonlea school and found all her pupils eager for work once more. Especially did the Queen's class gird up their loins for the fray, for at the end of the coming year, dimly shadowing their pathway already, loomed up that fateful thing known as »the Entrance,« at the thought of which one and all felt their hearts sink into their very shoes. Suppose they did not pass! That thought was doomed to haunt Anne through the waking hours of that winter, Sunday afternoons inclusive, to the almost entire exclusion of moral and theological problems. When Anne had bad dreams she found herself staring miserably at pass lists of the Entrance exams, where Gilbert Blythe's name was blazoned at the top and in which hers did not appear at all.

But it was a jolly, busy, happy swift-flying winter. Schoolwork was as interesting, class rivalry as absorbing, as of yore. New worlds of thought, feeling, and ambition, fresh, fascinating fields of unexplored knowledge seemed to be opening out before Anne's eager eyes.

»Hills peeped o'er hill and Alps on Alps arose.«

Much of all this was due to Miss Stacy's tactful, careful, broadminded guidance. She led her class to think and explore and discover for themselves and encouraged straying from the old beaten paths to a degree that quite shocked Mrs. Lynde and the school trustees, who viewed all innovations on established methods rather dubiously.

Apart from her studies Anne expanded socially, for Marilla, mindful of the Spencervale doctor's dictum, no longer vetoed occasional outings. The Debating Club flourished and gave several concerts; there were one or two parties almost verging on grown-up affairs; there were sleigh drives and skating frolics galore.

Between times Anne grew, shooting up so rapidly that Marilla was astonished one day, when they were standing side by side, to find the girl was taller than herself.

»Why, Anne, how you've grown!« she said, almost unbelievingly. A sigh followed on the words. Marilla felt a queer regret over Anne's inches. The child she had learned to love had vanished somehow and here was this tall, serious-eyed girl of fifteen, with the thoughtful brows and the proudly poised little head, in her place. Marilla loved the girl as much as she had loved the child, but she was conscious of a queer sorrowful sense of loss. And that night, when Anne had gone to prayer meeting with Diana, Marilla sat alone in the wintry twilight and indulged in the weakness of a cry. Matthew, coming in with a lantern, caught her at it and gazed at her in such consternation that Marilla had to laugh through her tears.

»I was thinking about Anne,« she explained. »She's got to be such a big girl—and she'll probably be away from us next winter. I'll miss her terrible.«

»She'll be able to come home often,« comforted Matthew, to whom Anne was as yet and always would be the little, eager girl he had brought home from Bright River on that June evening four years before. »The branch railroad will be built to Carmody by that time.«

»It won't be the same thing as having her here all the time,« sighed Marilla gloomily, determined to enjoy her luxury of grief uncomforted. »But there—men can't understand these things!«

There were other changes in Anne no less real than the physical change. For one thing, she became much quieter. Perhaps she thought all the more and dreamed as much as ever, but she certainly talked less. Marilla noticed and commented on this also.

»You don't chatter half as much as you used to, Anne, nor use half as many big words. What has come over you?«

Anne coloured and laughed a little, as she dropped her book and looked dreamily out of the window, where big fat red buds were bursting out on the creeper in response to the lure of the spring sunshine.

»I don't know—I don't want to talk as much,« she said, denting her chin thoughtfully with her forefinger. »It's nicer to think dear, pretty thoughts and keep them in one's heart, like treasures. I don't like to have them laughed at or wondered over. And somehow I don't want to use big words any more. It's almost a pity, isn't it, now that I'm really growing big enough to say them if I did want to. It's fun to be almost grown up in some ways, but it's not the kind of fun I expected, Marilla. There's so much to learn and do and think that there isn't time for big words. Besides, Miss Stacy says the short ones are much stronger and better. She makes us write all our essays as simply as possible. It was hard at first. I was so used to crowding in all the fine big words I could think of—and I thought of any number of them. But I've got used to it now and I see it's so much better.«

»What has become of your story club? I haven't heard you speak of it for a long time.«

»The story club isn't in existence any longer. We hadn't time for it—and anyhow I think we had got tired of it. It was silly to be writing about love and murder and elopements and mysteries. Miss Stacy sometimes has us write a story for training in composition, but she won't let us write anything but what might happen in Avonlea in our own lives, and she criticizes it very sharply and makes us criticize our own too. I never thought my compositions had so many faults until I began to look for them myself. I felt so ashamed I wanted to give up altogether, but Miss Stacy said I could learn to write well if I only trained myself to be my own severest critic. And so I am trying to.«

»You've only two more months before the Entrance,« said Marilla. »Do you think you'll be able to get through?«

Anne shivered.

»I don't know. Sometimes I think I'll be all right—and then I get horribly afraid. We've studied hard and Miss Stacy has drilled us thoroughly, but we mayn't get through for all that. We've each got a stumbling block. Mine is geometry of course, and Jane's is Latin, and Ruby and Charlie's is algebra, and Josie's is arithmetic. Moody Spurgeon says he feels it in his bones that he is going to fail in English history. Miss Stacy is going to give us examinations in June just as hard as we'll have at the Entrance and mark us just as strictly, so we'll have some idea. I wish it was all over, Marilla. It haunts me. Sometimes I wake up in the night and wonder what I'll do if I don't pass.«

»Why, go to school next year and try again,« said Marilla unconcernedly.

»Oh, I don't believe I'd have the heart for it. It would be such a disgrace to fail, especially if Gil—if the others passed. And I get so nervous in an examination that I'm likely to make a mess of it. I wish I had nerves like Jane Andrews. Nothing rattles her.«

Anne sighed and, dragging her eyes from the witcheries of the spring world, the beckoning day of breeze and blue, and the green things upspringing in the garden, buried herself resolutely in her book. There would be other springs, but if she did not succeed in passing the Entrance, Anne felt convinced that she would never recover sufficiently to enjoy them.