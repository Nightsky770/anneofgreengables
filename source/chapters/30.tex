%!TeX root=../annetop.tex
\chapter{The Queens Class is Organized}

\lettrine[lines=4]{M}{arilla} laid her knitting on her lap and leaned back in her chair. Her eyes were tired, and she thought vaguely that she must see about having her glasses changed the next time she went to town, for her eyes had grown tired very often of late.

It was nearly dark, for the full November twilight had fallen around Green Gables, and the only light in the kitchen came from the dancing red flames in the stove.

Anne was curled up Turk-fashion on the hearthrug, gazing into that joyous glow where the sunshine of a hundred summers was being distilled from the maple cordwood. She had been reading, but her book had slipped to the floor, and now she was dreaming, with a smile on her parted lips. Glittering castles in Spain were shaping themselves out of the mists and rainbows of her lively fancy; adventures wonderful and enthralling were happening to her in cloudland—adventures that always turned out triumphantly and never involved her in scrapes like those of actual life.

Marilla looked at her with a tenderness that would never have been suffered to reveal itself in any clearer light than that soft mingling of fireshine and shadow. The lesson of a love that should display itself easily in spoken word and open look was one Marilla could never learn. But she had learned to love this slim, gray-eyed girl with an affection all the deeper and stronger from its very undemonstrativeness. Her love made her afraid of being unduly indulgent, indeed. She had an uneasy feeling that it was rather sinful to set one's heart so intensely on any human creature as she had set hers on Anne, and perhaps she performed a sort of unconscious penance for this by being stricter and more critical than if the girl had been less dear to her. Certainly Anne herself had no idea how Marilla loved her. She sometimes thought wistfully that Marilla was very hard to please and distinctly lacking in sympathy and understanding. But she always checked the thought reproachfully, remembering what she owed to Marilla.

»Anne,« said Marilla abruptly, »Miss Stacy was here this afternoon when you were out with Diana.«

Anne came back from her other world with a start and a sigh.

»Was she? Oh, I'm so sorry I wasn't in. Why didn't you call me, Marilla? Diana and I were only over in the Haunted Wood. It's lovely in the woods now. All the little wood things—the ferns and the satin leaves and the crackerberries—have gone to sleep, just as if somebody had tucked them away until spring under a blanket of leaves. I think it was a little gray fairy with a rainbow scarf that came tiptoeing along the last moonlight night and did it. Diana wouldn't say much about that, though. Diana has never forgotten the scolding her mother gave her about imagining ghosts into the Haunted Wood. It had a very bad effect on Diana's imagination. It blighted it. Mrs.~Lynde says Myrtle Bell is a blighted being. I asked Ruby Gillis why Myrtle was blighted, and Ruby said she guessed it was because her young man had gone back on her. Ruby Gillis thinks of nothing but young men, and the older she gets the worse she is. Young men are all very well in their place, but it doesn't do to drag them into everything, does it? Diana and I are thinking seriously of promising each other that we will never marry but be nice old maids and live together forever. Diana hasn't quite made up her mind though, because she thinks perhaps it would be nobler to marry some wild, dashing, wicked young man and reform him. Diana and I talk a great deal about serious subjects now, you know. We feel that we are so much older than we used to be that it isn't becoming to talk of childish matters. It's such a solemn thing to be almost fourteen, Marilla. Miss Stacy took all us girls who are in our teens down to the brook last Wednesday, and talked to us about it. She said we couldn't be too careful what habits we formed and what ideals we acquired in our teens, because by the time we were twenty our characters would be developed and the foundation laid for our whole future life. And she said if the foundation was shaky we could never build anything really worth while on it. Diana and I talked the matter over coming home from school. We felt extremely solemn, Marilla. And we decided that we would try to be very careful indeed and form respectable habits and learn all we could and be as sensible as possible, so that by the time we were twenty our characters would be properly developed. It's perfectly appalling to think of being twenty, Marilla. It sounds so fearfully old and grown up. But why was Miss Stacy here this afternoon?«

»That is what I want to tell you, Anne, if you'll ever give me a chance to get a word in edgewise. She was talking about you.«

»About me?« Anne looked rather scared. Then she flushed and exclaimed:

»Oh, I know what she was saying. I meant to tell you, Marilla, honestly I did, but I forgot. Miss Stacy caught me reading Ben Hur in school yesterday afternoon when I should have been studying my Canadian history. Jane Andrews lent it to me. I was reading it at dinner hour, and I had just got to the chariot race when school went in. I was simply wild to know how it turned out—although I felt sure Ben Hur must win, because it wouldn't be poetical justice if he didn't—so I spread the history open on my desk lid and then tucked Ben Hur between the desk and my knee. I just looked as if I were studying Canadian history, you know, while all the while I was revelling in Ben Hur. I was so interested in it that I never noticed Miss Stacy coming down the aisle until all at once I just looked up and there she was looking down at me, so reproachful-like. I can't tell you how ashamed I felt, Marilla, especially when I heard Josie Pye giggling. Miss Stacy took Ben Hur away, but she never said a word then. She kept me in at recess and talked to me. She said I had done very wrong in two respects. First, I was wasting the time I ought to have put on my studies; and secondly, I was deceiving my teacher in trying to make it appear I was reading a history when it was a storybook instead. I had never realized until that moment, Marilla, that what I was doing was deceitful. I was shocked. I cried bitterly, and asked Miss Stacy to forgive me and I'd never do such a thing again; and I offered to do penance by never so much as looking at Ben Hur for a whole week, not even to see how the chariot race turned out. But Miss Stacy said she wouldn't require that, and she forgave me freely. So I think it wasn't very kind of her to come up here to you about it after all.«

»Miss Stacy never mentioned such a thing to me, Anne, and its only your guilty conscience that's the matter with you. You have no business to be taking storybooks to school. You read too many novels anyhow. When I was a girl I wasn't so much as allowed to look at a novel.«

»Oh, how can you call Ben Hur a novel when it's really such a religious book?« protested Anne. »Of course it's a little too exciting to be proper reading for Sunday, and I only read it on weekdays. And I never read any book now unless either Miss Stacy or Mrs.~Allan thinks it is a proper book for a girl thirteen and three-quarters to read. Miss Stacy made me promise that. She found me reading a book one day called, The Lurid Mystery of the Haunted Hall. It was one Ruby Gillis had lent me, and, oh, Marilla, it was so fascinating and creepy. It just curdled the blood in my veins. But Miss Stacy said it was a very silly, unwholesome book, and she asked me not to read any more of it or any like it. I didn't mind promising not to read any more like it, but it was agonizing to give back that book without knowing how it turned out. But my love for Miss Stacy stood the test and I did. It's really wonderful, Marilla, what you can do when you're truly anxious to please a certain person.«

»Well, I guess I'll light the lamp and get to work,« said Marilla. »I see plainly that you don't want to hear what Miss Stacy had to say. You're more interested in the sound of your own tongue than in anything else.«

»Oh, indeed, Marilla, I do want to hear it,« cried Anne contritely. »I won't say another word—not one. I know I talk too much, but I am really trying to overcome it, and although I say far too much, yet if you only knew how many things I want to say and don't, you'd give me some credit for it. Please tell me, Marilla.«

»Well, Miss Stacy wants to organize a class among her advanced students who mean to study for the entrance examination into Queen's. She intends to give them extra lessons for an hour after school. And she came to ask Matthew and me if we would like to have you join it. What do you think about it yourself, Anne? Would you like to go to Queen's and pass for a teacher?«

»Oh, Marilla!« Anne straightened to her knees and clasped her hands. »It's been the dream of my life—that is, for the last six months, ever since Ruby and Jane began to talk of studying for the Entrance. But I didn't say anything about it, because I supposed it would be perfectly useless. I'd love to be a teacher. But won't it be dreadfully expensive? Mr.~Andrews says it cost him one hundred and fifty dollars to put Prissy through, and Prissy wasn't a dunce in geometry.«

»I guess you needn't worry about that part of it. When Matthew and I took you to bring up we resolved we would do the best we could for you and give you a good education. I believe in a girl being fitted to earn her own living whether she ever has to or not. You'll always have a home at Green Gables as long as Matthew and I are here, but nobody knows what is going to happen in this uncertain world, and it's just as well to be prepared. So you can join the Queen's class if you like, Anne.«

»Oh, Marilla, thank you.« Anne flung her arms about Marilla's waist and looked up earnestly into her face. »I'm extremely grateful to you and Matthew. And I'll study as hard as I can and do my very best to be a credit to you. I warn you not to expect much in geometry, but I think I can hold my own in anything else if I work hard.«

»I dare say you'll get along well enough. Miss Stacy says you are bright and diligent.« Not for worlds would Marilla have told Anne just what Miss Stacy had said about her; that would have been to pamper vanity. »You needn't rush to any extreme of killing yourself over your books. There is no hurry. You won't be ready to try the Entrance for a year and a half yet. But it's well to begin in time and be thoroughly grounded, Miss Stacy says.«

»I shall take more interest than ever in my studies now,« said Anne blissfully, »because I have a purpose in life. Mr.~Allan says everybody should have a purpose in life and pursue it faithfully. Only he says we must first make sure that it is a worthy purpose. I would call it a worthy purpose to want to be a teacher like Miss Stacy, wouldn't you, Marilla? I think it's a very noble profession.«

The Queen's class was organized in due time. Gilbert Blythe, Anne Shirley, Ruby Gillis, Jane Andrews, Josie Pye, Charlie Sloane, and Moody Spurgeon MacPherson joined it. Diana Barry did not, as her parents did not intend to send her to Queen's. This seemed nothing short of a calamity to Anne. Never, since the night on which Minnie May had had the croup, had she and Diana been separated in anything. On the evening when the Queen's class first remained in school for the extra lessons and Anne saw Diana go slowly out with the others, to walk home alone through the Birch Path and Violet Vale, it was all the former could do to keep her seat and refrain from rushing impulsively after her chum. A lump came into her throat, and she hastily retired behind the pages of her uplifted Latin grammar to hide the tears in her eyes. Not for worlds would Anne have had Gilbert Blythe or Josie Pye see those tears.

»But, oh, Marilla, I really felt that I had tasted the bitterness of death, as Mr.~Allan said in his sermon last Sunday, when I saw Diana go out alone,« she said mournfully that night. »I thought how splendid it would have been if Diana had only been going to study for the Entrance, too. But we can't have things perfect in this imperfect world, as Mrs.~Lynde says. Mrs.~Lynde isn't exactly a comforting person sometimes, but there's no doubt she says a great many very true things. And I think the Queen's class is going to be extremely interesting. Jane and Ruby are just going to study to be teachers. That is the height of their ambition. Ruby says she will only teach for two years after she gets through, and then she intends to be married. Jane says she will devote her whole life to teaching, and never, never marry, because you are paid a salary for teaching, but a husband won't pay you anything, and growls if you ask for a share in the egg and butter money. I expect Jane speaks from mournful experience, for Mrs.~Lynde says that her father is a perfect old crank, and meaner than second skimmings. Josie Pye says she is just going to college for education's sake, because she won't have to earn her own living; she says of course it is different with orphans who are living on charity—they have to hustle. Moody Spurgeon is going to be a minister. Mrs.~Lynde says he couldn't be anything else with a name like that to live up to. I hope it isn't wicked of me, Marilla, but really the thought of Moody Spurgeon being a minister makes me laugh. He's such a funny-looking boy with that big fat face, and his little blue eyes, and his ears sticking out like flaps. But perhaps he will be more intellectual looking when he grows up. Charlie Sloane says he's going to go into politics and be a member of Parliament, but Mrs.~Lynde says he'll never succeed at that, because the Sloanes are all honest people, and it's only rascals that get on in politics nowadays.«

»What is Gilbert Blythe going to be?« queried Marilla, seeing that Anne was opening her Cæsar.

»I don't happen to know what Gilbert Blythe's ambition in life is—if he has any,« said Anne scornfully.

There was open rivalry between Gilbert and Anne now. Previously the rivalry had been rather one-sided, but there was no longer any doubt that Gilbert was as determined to be first in class as Anne was. He was a foeman worthy of her steel. The other members of the class tacitly acknowledged their superiority, and never dreamed of trying to compete with them.

Since the day by the pond when she had refused to listen to his plea for forgiveness, Gilbert, save for the aforesaid determined rivalry, had evinced no recognition whatever of the existence of Anne Shirley. He talked and jested with the other girls, exchanged books and puzzles with them, discussed lessons and plans, sometimes walked home with one or the other of them from prayer meeting or Debating Club. But Anne Shirley he simply ignored, and Anne found out that it is not pleasant to be ignored. It was in vain that she told herself with a toss of her head that she did not care. Deep down in her wayward, feminine little heart she knew that she did care, and that if she had that chance of the Lake of Shining Waters again she would answer very differently. All at once, as it seemed, and to her secret dismay, she found that the old resentment she had cherished against him was gone—gone just when she most needed its sustaining power. It was in vain that she recalled every incident and emotion of that memorable occasion and tried to feel the old satisfying anger. That day by the pond had witnessed its last spasmodic flicker. Anne realized that she had forgiven and forgotten without knowing it. But it was too late.

And at least neither Gilbert nor anybody else, not even Diana, should ever suspect how sorry she was and how much she wished she hadn't been so proud and horrid! She determined to »shroud her feelings in deepest oblivion,« and it may be stated here and now that she did it, so successfully that Gilbert, who possibly was not quite so indifferent as he seemed, could not console himself with any belief that Anne felt his retaliatory scorn. The only poor comfort he had was that she snubbed Charlie Sloane, unmercifully, continually, and undeservedly.

Otherwise the winter passed away in a round of pleasant duties and studies. For Anne the days slipped by like golden beads on the necklace of the year. She was happy, eager, interested; there were lessons to be learned and honour to be won; delightful books to read; new pieces to be practised for the Sunday-school choir; pleasant Saturday afternoons at the manse with Mrs.~Allan; and then, almost before Anne realized it, spring had come again to Green Gables and all the world was abloom once more.

Studies palled just a wee bit then; the Queen's class, left behind in school while the others scattered to green lanes and leafy wood cuts and meadow byways, looked wistfully out of the windows and discovered that Latin verbs and French exercises had somehow lost the tang and zest they had possessed in the crisp winter months. Even Anne and Gilbert lagged and grew indifferent. Teacher and taught were alike glad when the term was ended and the glad vacation days stretched rosily before them.

»But you've done good work this past year,« Miss Stacy told them on the last evening, »and you deserve a good, jolly vacation. Have the best time you can in the out-of-door world and lay in a good stock of health and vitality and ambition to carry you through next year. It will be the tug of war, you know—the last year before the Entrance.«

»Are you going to be back next year, Miss Stacy?« asked Josie Pye.

Josie Pye never scrupled to ask questions; in this instance the rest of the class felt grateful to her; none of them would have dared to ask it of Miss Stacy, but all wanted to, for there had been alarming rumours running at large through the school for some time that Miss Stacy was not coming back the next year—that she had been offered a position in the grade school of her own home district and meant to accept. The Queen's class listened in breathless suspense for her answer.

»Yes, I think I will,« said Miss Stacy. »I thought of taking another school, but I have decided to come back to Avonlea. To tell the truth, I've grown so interested in my pupils here that I found I couldn't leave them. So I'll stay and see you through.«

»Hurrah!« said Moody Spurgeon. Moody Spurgeon had never been so carried away by his feelings before, and he blushed uncomfortably every time he thought about it for a week.

»Oh, I'm so glad,« said Anne, with shining eyes. »Dear Stacy, it would be perfectly dreadful if you didn't come back. I don't believe I could have the heart to go on with my studies at all if another teacher came here.«

When Anne got home that night she stacked all her textbooks away in an old trunk in the attic, locked it, and threw the key into the blanket box.

»I'm not even going to look at a schoolbook in vacation,« she told Marilla. »I've studied as hard all the term as I possibly could and I've pored over that geometry until I know every proposition in the first book off by heart, even when the letters are changed. I just feel tired of everything sensible and I'm going to let my imagination run riot for the summer. Oh, you needn't be alarmed, Marilla. I'll only let it run riot within reasonable limits. But I want to have a real good jolly time this summer, for maybe it's the last summer I'll be a little girl. Mrs.~Lynde says that if I keep stretching out next year as I've done this I'll have to put on longer skirts. She says I'm all running to legs and eyes. And when I put on longer skirts I shall feel that I have to live up to them and be very dignified. It won't even do to believe in fairies then, I'm afraid; so I'm going to believe in them with all my whole heart this summer. I think we're going to have a very gay vacation. Ruby Gillis is going to have a birthday party soon and there's the Sunday school picnic and the missionary concert next month. And Mr.~Barry says that some evening he'll take Diana and me over to the White Sands Hotel and have dinner there. They have dinner there in the evening, you know. Jane Andrews was over once last summer and she says it was a dazzling sight to see the electric lights and the flowers and all the lady guests in such beautiful dresses. Jane says it was her first glimpse into high life and she'll never forget it to her dying day.«

Mrs.~Lynde came up the next afternoon to find out why Marilla had not been at the Aid meeting on Thursday. When Marilla was not at Aid meeting people knew there was something wrong at Green Gables.

»Matthew had a bad spell with his heart Thursday,« Marilla explained, »and I didn't feel like leaving him. Oh, yes, he's all right again now, but he takes them spells oftener than he used to and I'm anxious about him. The doctor says he must be careful to avoid excitement. That's easy enough, for Matthew doesn't go about looking for excitement by any means and never did, but he's not to do any very heavy work either and you might as well tell Matthew not to breathe as not to work. Come and lay off your things, Rachel. You'll stay to tea?«

»Well, seeing you're so pressing, perhaps I might as well, stay« said Mrs.~Rachel, who had not the slightest intention of doing anything else.

Mrs.~Rachel and Marilla sat comfortably in the parlour while Anne got the tea and made hot biscuits that were light and white enough to defy even Mrs.~Rachel's criticism.

»I must say Anne has turned out a real smart girl,« admitted Mrs.~Rachel, as Marilla accompanied her to the end of the lane at sunset. »She must be a great help to you.«

»She is,« said Marilla, »and she's real steady and reliable now. I used to be afraid she'd never get over her featherbrained ways, but she has and I wouldn't be afraid to trust her in anything now.«

»I never would have thought she'd have turned out so well that first day I was here three years ago,« said Mrs.~Rachel. »Lawful heart, shall I ever forget that tantrum of hers! When I went home that night I says to Thomas, says I, »Mark my words, Thomas, Marilla Cuthbert `ll live to rue the step she's took.« But I was mistaken and I'm real glad of it. I ain't one of those kind of people, Marilla, as can never be brought to own up that they've made a mistake. No, that never was my way, thank goodness. I did make a mistake in judging Anne, but it weren't no wonder, for an odder, unexpecteder witch of a child there never was in this world, that's what. There was no ciphering her out by the rules that worked with other children. It's nothing short of wonderful how she's improved these three years, but especially in looks. She's a real pretty girl got to be, though I can't say I'm overly partial to that pale, big-eyed style myself. I like more snap and colour, like Diana Barry has or Ruby Gillis. Ruby Gillis's looks are real showy. But somehow—I don't know how it is but when Anne and them are together, though she ain't half as handsome, she makes them look kind of common and overdone—something like them white June lilies she calls narcissus alongside of the big, red peonies, that's what.«