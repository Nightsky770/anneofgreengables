%!TeX root=../annetop.tex
\chapter{The Winter at Queen’s}

\lettrine[lines=4]{A}{nne's} homesickness wore off, greatly helped in the wearing by her weekend visits home. As long as the open weather lasted the Avonlea students went out to Carmody on the new branch railway every Friday night. Diana and several other Avonlea young folks were generally on hand to meet them and they all walked over to Avonlea in a merry party. Anne thought those Friday evening gypsyings over the autumnal hills in the crisp golden air, with the homelights of Avonlea twinkling beyond, were the best and dearest hours in the whole week.

Gilbert Blythe nearly always walked with Ruby Gillis and carried her satchel for her. Ruby was a very handsome young lady, now thinking herself quite as grown up as she really was; she wore her skirts as long as her mother would let her and did her hair up in town, though she had to take it down when she went home. She had large, bright-blue eyes, a brilliant complexion, and a plump showy figure. She laughed a great deal, was cheerful and good-tempered, and enjoyed the pleasant things of life frankly.

»But I shouldn’t think she was the sort of girl Gilbert would like,« whispered Jane to Anne. Anne did not think so either, but she would not have said so for the Avery scholarship. She could not help thinking, too, that it would be very pleasant to have such a friend as Gilbert to jest and chatter with and exchange ideas about books and studies and ambitions. Gilbert had ambitions, she knew, and Ruby Gillis did not seem the sort of person with whom such could be profitably discussed.

There was no silly sentiment in Anne’s ideas concerning Gilbert. Boys were to her, when she thought about them at all, merely possible good comrades. If she and Gilbert had been friends she would not have cared how many other friends he had nor with whom he walked. She had a genius for friendship; girl friends she had in plenty; but she had a vague consciousness that masculine friendship might also be a good thing to round out one’s conceptions of companionship and furnish broader standpoints of judgment and comparison. Not that Anne could have put her feelings on the matter into just such clear definition. But she thought that if Gilbert had ever walked home with her from the train, over the crisp fields and along the ferny byways, they might have had many and merry and interesting conversations about the new world that was opening around them and their hopes and ambitions therein. Gilbert was a clever young fellow, with his own thoughts about things and a determination to get the best out of life and put the best into it. Ruby Gillis told Jane Andrews that she didn’t understand half the things Gilbert Blythe said; he talked just like Anne Shirley did when she had a thoughtful fit on and for her part she didn’t think it any fun to be bothering about books and that sort of thing when you didn’t have to. Frank Stockley had lots more dash and go, but then he wasn’t half as good-looking as Gilbert and she really couldn’t decide which she liked best!

In the Academy Anne gradually drew a little circle of friends about her, thoughtful, imaginative, ambitious students like herself. With the »rose-red« girl, Stella Maynard, and the »dream girl,« Priscilla Grant, she soon became intimate, finding the latter pale spiritual-looking maiden to be full to the brim of mischief and pranks and fun, while the vivid, black-eyed Stella had a heartful of wistful dreams and fancies, as aerial and rainbow-like as Anne’s own.

After the Christmas holidays the Avonlea students gave up going home on Fridays and settled down to hard work. By this time all the Queen’s scholars had gravitated into their own places in the ranks and the various classes had assumed distinct and settled shadings of individuality. Certain facts had become generally accepted. It was admitted that the medal contestants had practically narrowed down to three—Gilbert Blythe, Anne Shirley, and Lewis Wilson; the Avery scholarship was more doubtful, any one of a certain six being a possible winner. The bronze medal for mathematics was considered as good as won by a fat, funny little up-country boy with a bumpy forehead and a patched coat.

Ruby Gillis was the handsomest girl of the year at the Academy; in the Second Year classes Stella Maynard carried off the palm for beauty, with small but critical minority in favour of Anne Shirley. Ethel Marr was admitted by all competent judges to have the most stylish modes of hair-dressing, and Jane Andrews—plain, plodding, conscientious Jane—carried off the honours in the domestic science course. Even Josie Pye attained a certain pre-eminence as the sharpest-tongued young lady in attendance at Queen’s. So it may be fairly stated that Miss Stacy’s old pupils held their own in the wider arena of the academical course.

Anne worked hard and steadily. Her rivalry with Gilbert was as intense as it had ever been in Avonlea school, although it was not known in the class at large, but somehow the bitterness had gone out of it. Anne no longer wished to win for the sake of defeating Gilbert; rather, for the proud consciousness of a well-won victory over a worthy foeman. It would be worth while to win, but she no longer thought life would be insupportable if she did not.

In spite of lessons the students found opportunities for pleasant times. Anne spent many of her spare hours at Beechwood and generally ate her Sunday dinners there and went to church with Miss Barry. The latter was, as she admitted, growing old, but her black eyes were not dim nor the vigour of her tongue in the least abated. But she never sharpened the latter on Anne, who continued to be a prime favourite with the critical old lady.

»That Anne-girl improves all the time,« she said. »I get tired of other girls—there is such a provoking and eternal sameness about them. Anne has as many shades as a rainbow and every shade is the prettiest while it lasts. I don’t know that she is as amusing as she was when she was a child, but she makes me love her and I like people who make me love them. It saves me so much trouble in making myself love them.«

Then, almost before anybody realized it, spring had come; out in Avonlea the Mayflowers were peeping pinkly out on the sere barrens where snow-wreaths lingered; and the »mist of green« was on the woods and in the valleys. But in Charlottetown harassed Queen’s students thought and talked only of examinations.

»It doesn’t seem possible that the term is nearly over,« said Anne. »Why, last fall it seemed so long to look forward to—a whole winter of studies and classes. And here we are, with the exams looming up next week. Girls, sometimes I feel as if those exams meant everything, but when I look at the big buds swelling on those chestnut trees and the misty blue air at the end of the streets they don’t seem half so important.«

Jane and Ruby and Josie, who had dropped in, did not take this view of it. To them the coming examinations were constantly very important indeed—far more important than chestnut buds or Maytime hazes. It was all very well for Anne, who was sure of passing at least, to have her moments of belittling them, but when your whole future depended on them—as the girls truly thought theirs did—you could not regard them philosophically.

»I’ve lost seven pounds in the last two weeks,« sighed Jane. »It’s no use to say don’t worry. I will worry. Worrying helps you some—it seems as if you were doing something when you’re worrying. It would be dreadful if I failed to get my license after going to Queen’s all winter and spending so much money.«

»I don’t care,« said Josie Pye. »If I don’t pass this year I’m coming back next. My father can afford to send me. Anne, Frank Stockley says that Professor Tremaine said Gilbert Blythe was sure to get the medal and that Emily Clay would likely win the Avery scholarship.«

»That may make me feel badly tomorrow, Josie,« laughed Anne, »but just now I honestly feel that as long as I know the violets are coming out all purple down in the hollow below Green Gables and that little ferns are poking their heads up in Lovers’ Lane, it’s not a great deal of difference whether I win the Avery or not. I’ve done my best and I begin to understand what is meant by the »joy of the strife.« Next to trying and winning, the best thing is trying and failing. Girls, don’t talk about exams! Look at that arch of pale green sky over those houses and picture to yourself what it must look like over the purply-dark beech-woods back of Avonlea.«

»What are you going to wear for commencement, Jane?« asked Ruby practically.

Jane and Josie both answered at once and the chatter drifted into a side eddy of fashions. But Anne, with her elbows on the window sill, her soft cheek laid against her clasped hands, and her eyes filled with visions, looked out unheedingly across city roof and spire to that glorious dome of sunset sky and wove her dreams of a possible future from the golden tissue of youth’s own optimism. All the Beyond was hers with its possibilities lurking rosily in the oncoming years—each year a rose of promise to be woven into an immortal chaplet.