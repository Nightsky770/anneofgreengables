%!TeX root=../annetop.tex
\chapter{A Good Imagination Gone Wrong}

\lettrine[lines=4]{S}{pring} had come once more to Green Gables—the beautiful capricious, reluctant Canadian spring, lingering along through April and May in a succession of sweet, fresh, chilly days, with pink sunsets and miracles of resurrection and growth. The maples in Lover's Lane were red budded and little curly ferns pushed up around the Dryad's Bubble. Away up in the barrens, behind Mr.~Silas Sloane's place, the Mayflowers blossomed out, pink and white stars of sweetness under their brown leaves. All the school girls and boys had one golden afternoon gathering them, coming home in the clear, echoing twilight with arms and baskets full of flowery spoil.

»I'm so sorry for people who live in lands where there are no Mayflowers,« said Anne. »Diana says perhaps they have something better, but there couldn't be anything better than Mayflowers, could there, Marilla? And Diana says if they don't know what they are like they don't miss them. But I think that is the saddest thing of all. I think it would be tragic, Marilla, not to know what Mayflowers are like and not to miss them. Do you know what I think Mayflowers are, Marilla? I think they must be the souls of the flowers that died last summer and this is their heaven. But we had a splendid time today, Marilla. We had our lunch down in a big mossy hollow by an old well—such a romantic spot. Charlie Sloane dared Arty Gillis to jump over it, and Arty did because he wouldn't take a dare. Nobody would in school. It is very fashionable to dare. Mr.~Phillips gave all the Mayflowers he found to Prissy Andrews and I heard him to say »sweets to the sweet.« He got that out of a book, I know; but it shows he has some imagination. I was offered some Mayflowers too, but I rejected them with scorn. I can't tell you the person's name because I have vowed never to let it cross my lips. We made wreaths of the Mayflowers and put them on our hats; and when the time came to go home we marched in procession down the road, two by two, with our bouquets and wreaths, singing »My Home on the Hill.« Oh, it was so thrilling, Marilla. All Mr.~Silas Sloane's folks rushed out to see us and everybody we met on the road stopped and stared after us. We made a real sensation.«

»Not much wonder! Such silly doings!« was Marilla's response.

After the Mayflowers came the violets, and Violet Vale was empurpled with them. Anne walked through it on her way to school with reverent steps and worshipping eyes, as if she trod on holy ground.

»Somehow,« she told Diana, »when I'm going through here I don't really care whether Gil—whether anybody gets ahead of me in class or not. But when I'm up in school it's all different and I care as much as ever. There's such a lot of different Annes in me. I sometimes think that is why I'm such a troublesome person. If I was just the one Anne it would be ever so much more comfortable, but then it wouldn't be half so interesting.«

One June evening, when the orchards were pink blossomed again, when the frogs were singing silverly sweet in the marshes about the head of the Lake of Shining Waters, and the air was full of the savour of clover fields and balsamic fir woods, Anne was sitting by her gable window. She had been studying her lessons, but it had grown too dark to see the book, so she had fallen into wide-eyed reverie, looking out past the boughs of the Snow Queen, once more bestarred with its tufts of blossom.

In all essential respects the little gable chamber was unchanged. The walls were as white, the pincushion as hard, the chairs as stiffly and yellowly upright as ever. Yet the whole character of the room was altered. It was full of a new vital, pulsing personality that seemed to pervade it and to be quite independent of schoolgirl books and dresses and ribbons, and even of the cracked blue jug full of apple blossoms on the table. It was as if all the dreams, sleeping and waking, of its vivid occupant had taken a visible although unmaterial form and had tapestried the bare room with splendid filmy tissues of rainbow and moonshine. Presently Marilla came briskly in with some of Anne's freshly ironed school aprons. She hung them over a chair and sat down with a short sigh. She had had one of her headaches that afternoon, and although the pain had gone she felt weak and »tuckered out,« as she expressed it. Anne looked at her with eyes limpid with sympathy.

»I do truly wish I could have had the headache in your place, Marilla. I would have endured it joyfully for your sake.«

»I guess you did your part in attending to the work and letting me rest,« said Marilla. »You seem to have got on fairly well and made fewer mistakes than usual. Of course it wasn't exactly necessary to starch Matthew's handkerchiefs! And most people when they put a pie in the oven to warm up for dinner take it out and eat it when it gets hot instead of leaving it to be burned to a crisp. But that doesn't seem to be your way evidently.«

Headaches always left Marilla somewhat sarcastic.

»Oh, I'm so sorry,« said Anne penitently. »I never thought about that pie from the moment I put it in the oven till now, although I felt instinctively that there was something missing on the dinner table. I was firmly resolved, when you left me in charge this morning, not to imagine anything, but keep my thoughts on facts. I did pretty well until I put the pie in, and then an irresistible temptation came to me to imagine I was an enchanted princess shut up in a lonely tower with a handsome knight riding to my rescue on a coal-black steed. So that is how I came to forget the pie. I didn't know I starched the handkerchiefs. All the time I was ironing I was trying to think of a name for a new island Diana and I have discovered up the brook. It's the most ravishing spot, Marilla. There are two maple trees on it and the brook flows right around it. At last it struck me that it would be splendid to call it Victoria Island because we found it on the Queen's birthday. Both Diana and I are very loyal. But I'm sorry about that pie and the handkerchiefs. I wanted to be extra good today because it's an anniversary. Do you remember what happened this day last year, Marilla?«

»No, I can't think of anything special.«

»Oh, Marilla, it was the day I came to Green Gables. I shall never forget it. It was the turning point in my life. Of course it wouldn't seem so important to you. I've been here for a year and I've been so happy. Of course, I've had my troubles, but one can live down troubles. Are you sorry you kept me, Marilla?«

»No, I can't say I'm sorry,« said Marilla, who sometimes wondered how she could have lived before Anne came to Green Gables, »no, not exactly sorry. If you've finished your lessons, Anne, I want you to run over and ask Mrs.~Barry if she'll lend me Diana's apron pattern.«

»Oh—it's—it's too dark,« cried Anne.

»Too dark? Why, it's only twilight. And goodness knows you've gone over often enough after dark.«

»I'll go over early in the morning,« said Anne eagerly. »I'll get up at sunrise and go over, Marilla.«

»What has got into your head now, Anne Shirley? I want that pattern to cut out your new apron this evening. Go at once and be smart too.«

»I'll have to go around by the road, then,« said Anne, taking up her hat reluctantly.

»Go by the road and waste half an hour! I'd like to catch you!«

»I can't go through the Haunted Wood, Marilla,« cried Anne desperately.

Marilla stared.

»The Haunted Wood! Are you crazy? What under the canopy is the Haunted Wood?«

»The spruce wood over the brook,« said Anne in a whisper.

»Fiddlesticks! There is no such thing as a haunted wood anywhere. Who has been telling you such stuff?«

»Nobody,« confessed Anne. »Diana and I just imagined the wood was haunted. All the places around here are so—so—commonplace. We just got this up for our own amusement. We began it in April. A haunted wood is so very romantic, Marilla. We chose the spruce grove because it's so gloomy. Oh, we have imagined the most harrowing things. There's a white lady walks along the brook just about this time of the night and wrings her hands and utters wailing cries. She appears when there is to be a death in the family. And the ghost of a little murdered child haunts the corner up by Idlewild; it creeps up behind you and lays its cold fingers on your hand—so. Oh, Marilla, it gives me a shudder to think of it. And there's a headless man stalks up and down the path and skeletons glower at you between the boughs. Oh, Marilla, I wouldn't go through the Haunted Wood after dark now for anything. I'd be sure that white things would reach out from behind the trees and grab me.«

»Did ever anyone hear the like!« ejaculated Marilla, who had listened in dumb amazement. »Anne Shirley, do you mean to tell me you believe all that wicked nonsense of your own imagination?«

»Not believe exactly,« faltered Anne. »At least, I don't believe it in daylight. But after dark, Marilla, it's different. That is when ghosts walk.«

»There are no such things as ghosts, Anne.«

»Oh, but there are, Marilla,« cried Anne eagerly. »I know people who have seen them. And they are respectable people. Charlie Sloane says that his grandmother saw his grandfather driving home the cows one night after he'd been buried for a year. You know Charlie Sloane's grandmother wouldn't tell a story for anything. She's a very religious woman. And Mrs.~Thomas's father was pursued home one night by a lamb of fire with its head cut off hanging by a strip of skin. He said he knew it was the spirit of his brother and that it was a warning he would die within nine days. He didn't, but he died two years after, so you see it was really true. And Ruby Gillis says\longdash«

»Anne Shirley,« interrupted Marilla firmly, »I never want to hear you talking in this fashion again. I've had my doubts about that imagination of yours right along, and if this is going to be the outcome of it, I won't countenance any such doings. You'll go right over to Barry's, and you'll go through that spruce grove, just for a lesson and a warning to you. And never let me hear a word out of your head about haunted woods again.«

Anne might plead and cry as she liked—and did, for her terror was very real. Her imagination had run away with her and she held the spruce grove in mortal dread after nightfall. But Marilla was inexorable. She marched the shrinking ghost-seer down to the spring and ordered her to proceed straightaway over the bridge and into the dusky retreats of wailing ladies and headless spectres beyond.

»Oh, Marilla, how can you be so cruel?« sobbed Anne. »What would you feel like if a white thing did snatch me up and carry me off?«

»I'll risk it,« said Marilla unfeelingly. »You know I always mean what I say. I'll cure you of imagining ghosts into places. March, now.«

Anne marched. That is, she stumbled over the bridge and went shuddering up the horrible dim path beyond. Anne never forgot that walk. Bitterly did she repent the license she had given to her imagination. The goblins of her fancy lurked in every shadow about her, reaching out their cold, fleshless hands to grasp the terrified small girl who had called them into being. A white strip of birch bark blowing up from the hollow over the brown floor of the grove made her heart stand still. The long-drawn wail of two old boughs rubbing against each other brought out the perspiration in beads on her forehead. The swoop of bats in the darkness over her was as the wings of unearthly creatures. When she reached Mr.~William Bell's field she fled across it as if pursued by an army of white things, and arrived at the Barry kitchen door so out of breath that she could hardly gasp out her request for the apron pattern. Diana was away so that she had no excuse to linger. The dreadful return journey had to be faced. Anne went back over it with shut eyes, preferring to take the risk of dashing her brains out among the boughs to that of seeing a white thing. When she finally stumbled over the log bridge she drew one long shivering breath of relief.

»Well, so nothing caught you?« said Marilla unsympathetically.

»Oh, Mar—Marilla,« chattered Anne, »I'll b-b-be contt-tented with c-c-commonplace places after this.«